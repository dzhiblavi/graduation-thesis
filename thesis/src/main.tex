% !TEX program = xelatex
\documentclass[times,specification,annotation]{../common/itmo-student-thesis}

%% Опции пакета:
%% - specification - если есть, генерируется задание, иначе не генерируется
%% - annotation - если есть, генерируется аннотация, иначе не генерируется
%% - times - делает все шрифтом Times New Roman, собирается с помощью xelatex
%% - languages={...} - устанавливает перечень используемых языков. По умолчанию это {english,russian}.
%%                     Последний из языков определяет текст основного документа.

\usepackage{mathabx}
\usepackage{icomma}
\usepackage{tabularx}
\usepackage{tikz}
\usetikzlibrary{arrows}

\addbibresource{bibliography/common.bib}
\graphicspath{ {../assets/} }

\newcommand\tbd[1]{\textcolor{red}{#1}}

\begin{document}

\begin{center}
    \large{\textbf{\tbd{DRAFT~\today}}}
\end{center}

\studygroup{M3439}
\title{Разработка параллельных алгоритмов решения задачи булевой выполнимости с использованием 
       поиска вероятностных лазеек}
\author{Джиблави Ибрагим Билалович}{Джиблави И.Б.}
\supervisor{Чивилихин Даниил Сергеевич}{Чивилихин Д.С.}{к.т.н.}{ординарный доцент}
\publishyear{2022}
\startdate{31}{января}{2022}
\finishdate{15}{мая}{2022}
\defencedate{15}{июня}{2022}

\secretary{Павлова О.Н.}

%% Задание
%% Техническое задание и исходные данные к работе
\technicalspec{Ключевой задачей является разработка эффективного алгоритма решения задач булевой
    выполнимости. Алгоритм по заданной формуле должен возвращать информацию о её
    выполнимости. Подразумевается параллельный алгоритм, использующий поиск
    вероятностных лазеек для сведения задачи к набору более простых подзадач и
    последующему решению этих подзадач с применением существующих решателей.}

%% Содержание выпускной квалификационной работы (перечень подлежащих разработке вопросов)
\plannedcontents{В данной работе разработаны алгоритмы поиска вероятностных лазеек на основе
    эволюционных алгоритмов. Разработаны различные подходы к решению на основе методов
    поиска вероятностных лазеек, исследована их эффективность. Проведено сравнение
    эффективности с существующими конкурентоспособными алгоритмами.}

%% Исходные материалы и пособия
\plannedsources{
    \begin{enumerate}
        \item \tbd{ресурсы из ИЗ}.
    \end{enumerate}
}

%% Цель исследования
\researchaim{\tbd{цель}}

%% Задачи, решаемые в ВКР
\researchtargets{\begin{enumerate}
        \item разработка эффективных алгоритмов поиска вероятностных лазеек.
        \item разработка методов решения задачи булевой выполнимости с применением вероятностных
            лаззек.
        \item исследование эффективности полученных алгоритмов.
        \item \tbd{Здесь возможно надо более подробный план или вообще просто про "крутой решатель"?}
    \end{enumerate}}

%% Использование современных пакетов компьютерных программ и технологий
\addadvancedsoftware{\tbd{пакеты}}{\tbd{где}}

%% Краткая характеристика полученных результатов
\researchsummary{\tbd{оценка результатов}}

\researchfunding{\tbd{вроде не было}}

\researchpublications{\tbd{вроде не было, если не считать paper}}

\maketitle{Бакалавр}

\tableofcontents

%&latex
\label{introduction}
\startprefacepage

Задача булевой выполнимости является крайне важной NP-полной~\cite{bib:cook-levin} алгоритмической
задачей, так как к ней сводится большое число задач. Среди них:
\begin{itemize}
    \item Проверка моделей, использующийся для формальной верификации параллельных систем с
        конечным числом состояний~\cite{bib:use-mc}.
    \item Восстановление секретного ключа в алгоритме RSA~\cite{bib:use-rsa}.
    \item Разложение булевых матриц~\cite{bib:use-bma}, часто применяемое в рекоммендательных
        системах.
    \item 0-1 задача о рюкзаке (широко известная задача класса NP).
    \item Многие задачи планирования, например, one shop scheduling~\cite{bib:use-ojs}, сводятся к SAT.
    \item \tbd{Больше примеров?}.
\end{itemize}

Для задачи булевой выполнимости, как и для любой NP-полной задачи пока не известно алгоритма,
решающего её за полиномиальное от размера входа время. Однако, спектр применения этой задачи
достаточно широк, чтобы возникала практическая польза от разработки эффективных решателей.

Алгоритмы решения задачи булевой выполнимости глобально делятся на две категории. Первая ---
последовательные решатели, то есть работающие в одном потоке исполнения. Существующие эффективные
и широко распостраненные подходы к построению таких решателей более подробно описаны в разделе~
\ref{overview:sequential}. Вторая --- параллельные решатели, нацеленные на максимальную утилизацию доступных
ресурсов многоядерных и многопроцессорных машин для ускорения процесса решения. Более подробно
они описаны в разделе~\ref{overview:parallel}.

Целью данной работы является построение эффективного параллельного алгоритма решения задачи булевой
выполнимости на основе поиска вероятностных лазеек. В рамках исследований в \tbd{paper} данное семейство
подходов показало положительные результаты, однако остается достаточно большое пространство как для
построения более конкурентоспособной реализации, так и для расширения семейства подходов.

В рамках данной работы поставлены и выполнены следующие задачи:
\begin{itemize}
    \item Разработка эффективного алгоритма поиска вероятностных лазеек на основе подходов,
        изложенных в \tbd{paper}.
    \item Адаптация набора существующих решателей для использования в рамках разрабатываемых
        алгоритмов.
    \item Разработка эффективного параллельного сервиса для решения задач с подстановками,
        допускающий обмен знаниями между решателями.
    \item Разработка схем решения с использованием вероятностных лазеек, как предложенных в \tbd{paper},
        так и альтернативных.
    \item Исследование эффективности реализованных схем и сравнение с существующими
        паралелльными решателями.
\end{itemize}

В главе~\ref{overview} содержатся необходимые понятия и определения, детально описаны поставленные
в данной работе задачи, описан подробный план по их решению. Также рассмотрены существующие
подходы к решению задач булевой выполнимости, как параллельные, имеющие непосредсвенное отношение
к данной работе как конкуренты, так и последовательные, устройство которых необходимо понимать,
так как на определенные свойства таких решателей делается упор в устранении недостатков
предлагаемых алгоритмов.

В главе~\ref{arch} детально изложены предлагаемые подходы к решению подзадач, возникших при
разработке алгоритма. Описана схема поиска вероятностных лазеек, способ разделения задач булевой
выполнимости и его применение в данной работе. Изложен подход к параллельному решению подзадач,
указаны недостатки наивного подхода, предложено и обосновано решение данных недостатков. Далее
описана структура програмного кода разработанного приложения.

В главе~\ref{research} описан процесс оценки качества и производительности полученного семейства
алгоритмов. Предоставлены и разобраны результаты экспериментов как над частями алгоритма, так и
над получившимся решатетем в целом. Сделаны выводы относительно конкурентоспособности рассматриваемых
схем решения.

%\printannobibliography


%&latex
\chapter{Обзор}\label{overview}

\tbd{Содержание главы}

\section{Термины и понятия}\label{overview:terms}

\tbd{Переработать определения? Лазейки точно плохие}

\begin{definition}\label{overview:formula}
    \textbf{Булевой формулой} \footnote{https://ru.wikipedia.org/wiki/Булева\_формула} называется 
    формула логики высказываний, сорержащая логические переменные и пропозициональные связки
    $\wedge, \vee, \neg$. Множество логических переменных обозначается $V$.
\end{definition}

\begin{definition}\label{overview:function}
    \textbf{Булевой функцией} называется отображение $E\colon~ \mathcal{B}^n \to \mathcal{B}$,
    где $\mathcal{B} = \{\, 0, 1 \,\}$, а $n$ -- число различных переменных. Булева формула $F$ 
    задает булеву функцию $E_F$. Далее разница между этими понятиями для нас несущественна, поэтому
    всегда будет упоминаться булева функция $E$.
\end{definition}

\begin{definition}\label{overview:sat}
    \textbf{Задача булевой выполнимости (SAT)} --- проверить, существует ли $x \in \mathcal{B}^n$
    такой, что выполняется $E(x) = 1$.
\end{definition}

\begin{theorem}(Кук, Левин)
    \textit{Задача булевой выполнимости принадлежит классу NP-полных задач}~\cite{bib:cook-levin}.
\end{theorem}

\begin{definition}\label{overview:solver}
    \textbf{Решателем, алгоритмом A} называется алгоритм, принимающий на вход описание булевой формулы $C$
    и выдающий результат проверки булевой выполнимости соответствующей булевой функции --- $A(C)$.
    Возможны следующие результаты:
    \begin{itemize}
        \item \texttt{0}, функция невыполнима.
        \item \texttt{1}, функция выполнима. В этом случае также может быть возвращено
            удовлетворяющее функции назначение переменных $R$: $E(V \mid R) = 1$.
        \item \texttt{?}, решатель не смог решить задачу либо вследствие его неполноты, либо
            из-за нехватки ресурсов, таких как время или память.
    \end{itemize}
    Обозначим за $S(A)$ множество булевых функций, разрешимых алгоритмом $A$ за полиномиальное время.
\end{definition}

\begin{definition}\label{overview:assumption}
    \textbf{Означиванием, подстановкой} набора переменных $B \subseteq V$ называется отображение
    $\hat{b}\colon~ B \to \mathcal{B}$. Означивание можно применить к булевой функции $E$,
    результатом будет другая булева функция $E[B \mid \hat{b}] \colon \mathcal{B}^{|V| - |B|} \to
    \mathcal{B}$, возможно тождественная. Она получается из исходной путем частичной подстановки
    переменных, назначенных функцией $\hat{b}$. Множество всех означиваний множества переменных
    $B$ обозначается $\hat{B}$.
\end{definition}

\begin{definition}\label{overview:prop}
    \textbf{Вывод последствий, Unit Propagation (UP)} --- алгоритм, принимающий на вход формулу
    и назначение набора переменных. Возвращает спиков литералов (то есть значений переменых),
    выведенных из заданного назначения, или информацию о возникновении конфликта, то есть о
    невыполнимости формулы с заданным назначением. Эффективно реализован в \textsc{Minisat}~
    \cite{bib:minisat}.
\end{definition}

\begin{definition}\label{overview:backdoor}
    \textbf{Лазейкой} называется множество $B \subseteq V$ такое, что для любого из $2^{|B|}$
    означиваний $\hat{b}$ переменных из $B$ выполняется $E[B \mid \hat{b}] \equiv 0$.
    Ясно, что если для функции $E$ существует лазейка, то она невыполнима, то есть $E \equiv 0$.
\end{definition}

\begin{definition}\label{overview:rho-backdoor}
    \textbf{$\rho$-Лазейкой} называется множество $B \subseteq V$ такое, что выполняется
\[
    \left|\left\{\, \hat{b} \mid \hat{b} \in \hat{B},~ E[B \mid \hat{b}] \equiv 0 \,\right\}\right|
    \geqslant \rho \cdot 2^{|B|}.
\]
\end{definition}

\newcommand*{\prob}{\mathsf{Pr}}

\begin{definition}\label{overview:prob-backdoor}
    \textbf{$(\varepsilon, \delta)$ аппроксимация лазейки, вероятностная лазейка}. Зафиксируем
    $\varepsilon, \delta \in (0, 1)$, пожмножество переменных $B \subseteq V$ и алгоритм $A$, 
    и рассмотрим множество означиваний $\hat{B}$ в терминах $\rho$-лазейки.
    \begin{itemize}
        \item Введем на $\hat{B}$ равномерное распределение и зададим случайную величину
            \[
                \xi_B(\hat{b}) = \left[E[B \mid \hat{b}] \in S(A)\right].
            \]
            Ясно, что эта величина распределена по Бернулли с $p = \rho_B$.
        \item Тогда $B$ является $(\varepsilon, \delta)$ аппроксимацией лазейки, если
            \[
                \prob{\left[1 - \varepsilon \leqslant \frac{1}{N}\sum_{j=1}^{N}{\xi_j}\right]}
                \geqslant 1 - \delta,
            \]
            где $N \geqslant \frac{4 \ln{2/\delta}}{\varepsilon^2}$. Данное условие согласно
            теореме Чернова \tbd{paper} обеспечивает тот факт, что аппроксимация $\rho_B$,
            вычисляемая по формуле 
            \begin{equation}
                \hat{\rho}_B = \frac{1}{N} \sum_{j=1}^{N}{\xi_j}\label{overview:rho-hat},
            \end{equation}
            отклоняется от истинного значения $\rho_B$ не более, чем на $\varepsilon$ с вероятностью 
            не менее $1 - \delta$.
    \end{itemize}
\end{definition}

\startrelatedwork

\section{Последовательные решатели}\label{overview:sequential}

Понимание устройства распостраненных схем последовательных решателей играет крайне важную роль в
данной работе, так как они прямо или косвенно используются в качестве решателей подзадач,
возникающих в процессе работы описываемого алгоритма.

Опишем в общих чертах схему работы классических последовательных решателей. 
\textit{Алгоритм Дэвиса-Патнема-Логемана-Лавленда (DPLL)}~\cite{bib:dpll-1}~\cite{bib:dpll-2}
--- полный алгоритм для решения задачи булевой выполнимости, основанный на поиске с возвратом. 
Данный алгоритм заключается в разбиении задачи на подзадачи путем последовательного присвоения 
переменным булевых значений и дальнейшей проверкой на отсутствие конфликтов. Часто в данном
типе алгоритмов применяются дополнительные правила: \textit{резолюция} --- вывод значения переменных,
значения которых однозначно определяют значение дизъюнкта, \textit{исключение чистых переменных},
то есть переменных, входящих в формулу либо только с отрицанием, либо только без отрицания.

На основе алгоритма DPLL построен подход \textit{CDCL (Conflict-Driven Clause Learning, управляемое
конфликтами обучение дизъюнктам)}~\cite{bib:cdcl}. Данный подход используется во всех решателях, 
включенных в качестве компонентов в данной работе, так как является одним из самых распространенных и
эффективных. Подход расширяет семейство DPLL несколькими техниками, среди которых анализ структуры
конфликтов, запоминание конфликтных дизъюнктов, эвристики ветвления и другие. Тот факт, что
данные решатели аккумулируют определенный вид знаний, крайне важен для портфельных решателей,
описанных в разделе~\ref{overview:parallel} и также использующихся в данной работе.

\section{Параллельные решатели}\label{overview:parallel}

Параллельные решатели нацелены на ускорение процесса решения засчет увеличения утилизации
выделенных ресурсов, в частности и в особенности многоядерных и многопроцессорных систем. Рассмотрим
некоторые подходы к реализации таких алгоритмов.

\textit{Портфельные решатели} основаны на запуске диверсифицированного набора последовательных
решателей, часто основанных на алгоритме CDCL, позволяющем осуществлять обмен 
знаниями~\cite{bib:painless-sharing}. Типичными примерами таких решателей являются 
\textsc{hordesat}~\cite{bib:hordesat-portfolio} и \textsc{painless}~\cite{bib:painless}. 
Диверсификация решателей неизбежно приводит к выучиванию разных наборов дизъюнктов, ускоряя тем самым 
каждый из них по отдельности засчет обмена. В данной работе используется фреймворк 
\textsc{painless}, который предоставляет инструменты
в том числе для создания портфельного решателя. В частности, портфельный решатель 
\textsc{painless-mcomsps}, основанный на последовательном решателе \textsc{MapleCOMSPS}
~\cite{bib:maplecomsps}, является победителем параллельного 
трека Sat Competition~\footnote{https://satcompetition.github.io/2021/slides/ISC2021-fixed.pdf} и 
используется в предлагаемом алгоритме.

\tbd{Может вообще убрать? Никак не используется}
\textit{Алгоритмы локального поиска}~\cite{bib:local-search}. Данный подход основан на параллельном
изменении значений переменных, и в данной работе не используется.

\textit{Алгоритмы разделяй-и-властвуй}~\cite{bib:divide-and-conquer}. В отличие от портфельных 
решателей, данное семейство алгоритмов не занимается параллельным решением одной и той же задачи. 
Напротив, исходная задача разбивается на подзадачи тем или иным способом, после чего полученные 
подзадачи решаются параллельно. Пример такого решателя основан на упомянутом фреймворке \textsc{painless}
~\cite{bib:dac-painless}. Основной проблемой данного подхода является тот факт, что подпространства 
поиска очень часто имеют значительно разную сложность решения. В данной работе также будет произведена
попытка решить эту проблему: предлагаемый в данной работе алгоритм относится к данному семейству
решателей и является совершенно новым в том смысле, что на момент разработки не существует
полноценных конкурентоспособных алгоритмов, основанных на идее разделения задачи на подзадачи через
поиск вероятностных лазеек. Тем не менее, результаты, продемонстрированные в работе \tbd{paper},
дают основания полагать, что на основе этой идеи можно создать решатель, имеющий высокую
производительность.

\finishrelatedwork

\chapterconclusion

\tbd{Заключение}


\chapter{Теоретическое исследование и архитектура}\label{arch}

\section{Поиск вероятностных лазеек}\label{arch:rbs}

\subsection{Схема алгоритма}\label{arch:rbs:schema}

В данной главе описаны алгоритмы поиска вероятностных лазеек.
Для поиска вероятностных лазеек применяются эволюционные алгоритмы, а именно, (1 + 1) и (q + h)
алгоритмы~\cite{bib:ea},~\cite{bib:ga}. Для полного описания теоретической схемы поиска 
вероятностных лазеек необходимо определить особь, фитнес-функцию, а также операторы скрещивания, 
мутации и отбора.

\textbf{Особью} во всех реализованных схемах поиска вероятностных лазеек является битовая маска
$\overline{B}$, соответствующая множеству переменных $B$, включенных в лазейку.

\textbf{Фитнес функция}. В фитнес-функции используется аппроксимация значения $\rho$, формально 
определенная уравнением \ref{overview:rho-hat}, так как такой
подход позволяет вычислять его с достаточно высокой точностью, а главное --- быстро. В качестве
алгоритма $A$ используется \textit{вывод последствий, или unit propagation (UP)}, эффективно реализованный
в рамках решателя Minisat~\cite{bib:minisat}. Данный алгоритм с точки зрения решателя не является
полным, но имеет полиномиальное время работы.

Используется фитнес функция, описанная в TBD-REF. Её преимущество заключается в том, что она помимо 
максимизации $\hat{\rho}$-значения лазейки минимизирует её размер. Максимизация $\hat{\rho}$-значения 
достигается первой частью функции:
\[
    G_{C}\left(\overline{B}\right) = (1 - \hat{\rho}_B) \cdot 2^{\omega |V|},
\]
где $C$ --- формула \ref{overview:formula}, $V$ --- множество переменных в формуле, $w \in (0, 1]$
-- константа. Минимизация размера лазейки обеспечивается второй частью функции:
\[
    f_{C, \min{|B|}} = \hat{\rho}_B \cdot 2^{|B|}.
\]
Действительно, при относительно близких значениях $\rho$ большое влияние на функцию будет оказывать
размер лазейки $B$. Итоговая фитнес функция выглядит следующим образом:
\begin{equation}
    f_{C}\left(\overline{B}\right) = \hat{\rho}_B \cdot 2^{|B|} + (1 - \hat{\rho}_B) \cdot 2^{\omega |V|}.
    \label{arch:fitness}
\end{equation}

\textbf{Операторы}
В качестве основного оператора мутации был выбран зарекоммендовавший себя в TBD-REF оператор 
\textsc{Doerr}~\cite{bib:doerr}. В процессе разработки также использовался равномерный оператор
мутации, однако результаты, которые он показывал, стабильно хуже результатов \textsc{Doerr} на
всех примерах, поэтому он был отброшен, однако доступен в конфигурации. Реализованы операторы
одноточечного и двуточечного скрещивания. Генетический алгоритм $(\mu + \lambda)$ был
реализован аналогично схеме, предложенной в TBD-REF.

\subsection{Эффективный вывод последствий}\label{arch:rbs:prop}

Основным потребителем ресурсов в описанной схеме поиска вероятностных лазеек является алгоритм
вывода последствий, реализованный в Minisat~\cite{bib:minisat}. Поэтому оптимизация этого алгоритма
и разработка новых подходов является крайне важной задачей для достижения высокой производительности.

\textit{Выборка}\label{arch:rbs:prop:sampling}. Заметим, что для достаточно маленьких $B$ имеет
смысл производить полный перебор множества $\hat{B}$ при вычислении $\hat{\rho}$. Во-первых,
это обеспечит точное значение $\hat{\rho} = \rho$, во-вторых, реализация перебора всех подстановок
(то есть, перебора всех возможных назначений переменных из набора в $\mathcal{B}$, что эквивалентно
перебору всех чисел от $0$ до $2^{|B|} - 1$ в двоичной записи) точно будет работать не медленнее,
чем случайный перебор $2^{|B|}$ подстановок, так как не пользуется геренацией случайных чисел и
делает строго меньше операций засчет того, что не на каждом шаге модифицируются все значения подстановки.

Данное наблюдение создает необходимость в абстракции от типа выборки. Для этого был создан интерфейс
\textsc{Search}, а также следующие реализации:
\begin{itemize}
    \item \textsc{FullSearch} --- перебор всех возможных $\hat{b} \in \hat{B}$.
    \item \textsc{RandomSearch} --- перебор $N$ случайных подстановок из $\hat{B}$.
    \item \textsc{UniqueSearch} --- перебор $N$ уникальных случайных подстановок из $\hat{B}$.
    \item \textsc{CartesianSearch} --- перебор всех подстановок из декартового произведения
        нескольких наборов подстановок, необходимая для реализации некоторых стратегий 
        (см. Главу \ref{arch:solve}).
\end{itemize}

На рисунке \ref{arch:rbs:prop:search-img} представлена схема наследования интерфейса \textsc{Search}.
Далее в таблице \ref{arch:rbs:prop:search-def} описаны методы интерфейса.

\begin{figure}[H]
    \caption{Интерфейс \textsc{Search} и его реализации}
    \centering
    \includegraphics[width=\textwidth]{arch-search}
    \label{arch:rbs:prop:search-img}
\end{figure}

\begin{table}[H]
    \caption{Описание методов \textsc{Search}}\label{arch:rbs:prop:search-def}
    \centering
    \begin{tabularx}{\textwidth}{|*{3}{>{\centering\arraybackslash}X|}}\hline
        Метод & Параметры & Описание \\\hline
        \textsc{operator()} & -- & Возвращает текущий элемент выборки \\\hline
        \textsc{operator++} & -- & Переходит на следующий элемент выборки. Возвращает \textsc{false},
                                    если новый элемент -- последний \\\hline
        \textsc{size} & -- & Возвращает размер выборки \\\hline
        \textsc{split} & Число потоков, номер потока & Возвращает часть исходной выборки \\\hline
    \end{tabularx}
\end{table}

После получения выборки необходимо вызвать метод вывода последствий на всех элементах этой
выборки. Далее будут рассмотрены различные подходы к решению этой задачи: \textit{наивный
(последовательный), параллельный, вычисление $\rho$ через дерево поиска}.

\textit{Наивный подход}\label{arch:rbs:prop:naive}. Данный подход подразумевает последовательный
вызов метода вывода последствий. То есть, имея выборку $D$ и решатель $A$, значение $\rho$ 
аппроксимируется так, как показано в листинге \ref{arch:rbs:prop:naive-lst}.

\algdef{SE}[DOWHILE]{Do}{doWhile}{\algorithmicdo}[1]{\algorithmicwhile\ #1}%

\begin{algorithm}[H]
\caption{Наивное вычисление $\hat{\rho}$}\label{arch:rbs:prop:naive-lst}
\begin{algorithmic}
	\Function{calculate\_rho}{\textsc{A}, \textsc{D}}
        \State $S \leftarrow 0$
        \Do
            \State $\hat{b} \leftarrow \textsc{D()}$
            \If{\textsc{A($\hat{b}$)} = 0}
				\State $S \leftarrow S + 1$
			\EndIf
        \doWhile{\textsc{D++}}
        \State\Return $\frac{S}{N}$
	\EndFunction
\end{algorithmic}
\end{algorithm}

\textit{Параллельная обработка выборки}\label{arch:rbs:prop:par}. Данный подход разделяет выборку
на несколько выборок и обрабатывает их в разных потоках. Для этого используется метод \textsc{split}.
Псевдокод данного подхода представлен в листинге \ref{arch:rbs:prop:par-lst}.

\begin{algorithm}[H]
\caption{Параллельное вычисление $\hat{\rho}$}\label{arch:rbs:prop:par-lst}
\begin{algorithmic}
    \Function{calculate\_rho\_par}{$[\textsc{A}_i, i=\overline{1,M}]$, \textsc{D}}
        \State $S \leftarrow 0$
        \State $[\textsc{D}_i] \leftarrow \left[\textsc{D.split($M$, $j$),~ $j=\overline{1,M}$}\right]$
        \State\Return $\sum_{i}^{\textsc{parallel}}{\textsc{calculate\_rho($A_i$, $D_i$)}} / N$
	\EndFunction
\end{algorithmic}
\end{algorithm}

Данный подход с теоретической точки зрения крайне хорошо масштабируется, так как метод \textsc{split}
занимает несравнимо мало времени по сравнению с многократным вызовом метода вывода последствий. Практическое
сравнение последовательного и параллельного подходов приведено в Главе \ref{research:prop}.

\textit{Вычисление точного значения $\rho$ через дерево подстановок}\label{arch:rbs:prop:tree}.
Данный подход является новым и не имеет аналогов в известных мне решателях за ненадобностью. Однако,
идея, описанная далее, является крайне эффективной оптимизацией перебора $\hat{B}$, как показано в
Главе~\ref{research:prop}. Для вычисления точного значения $\rho$ необходимо посчитать число
подстановок $\hat{b} \in \hat{B}$, которые решаются заданным алгоритмом $A$. В данной работе, как
упомянуто ранее, в качестве этого алгоритма используется алгоритм вывода последствий. Результатом
работы этого алгоритма может быть либо информация о существовании конфликтующих подстановок переменных,
либо отсутствие дополнительной информации. Заметим, что если на каком-то префиксе $\hat{b}$ 
алгоритм сообщил о конфликте, то нет никакого смысла далее рассматривать подстановки с этим префиксом,
так как все они заведомо конфликтные. Можно просто прибавить к результату размер всего поддерева поиска.
Иллюстрация к данной идее приведена на рисунке \ref{arch:rbs:prop:tree-img}. Листинг, содержащий
псевдокод данного метода, приведен в приложении (листинг~\ref{append:rbs:prop:tree-lst}). Данный
метод был успешно реализован и встроен в исходный код решателя Minisat. Сравнительные результаты
с наивным и параллельным подходами содержатся в Главе~\ref{research:prop}.

\begin{figure}[H]
    \caption{Вычисление $\rho$ через дерево подстановок. Здесь $X,Y,Z$ --- переменные. Подстановки
    $E[X \mid 1]$ и $E[\{\,X,Y\,\} \mid \{\, X \to 0,~ Y \to 1 \,\}]$ приводят к конфликту, поэтому
    соответствующие поддеревья поиска можно не рассматривать.}
    \centering
    \includegraphics[width=\textwidth]{arch-tree}
    \label{arch:rbs:prop:tree-img}
\end{figure}

\section{Решение набора подстановок}\label{arch:solver}

В данной главе описана архитектура сервиса решения булевой функции с подстановками. Описан интерфейс
сервиса, приведена схема распределения задач по последовательным решателям, описана техника обмена
знаниями между решателями.

В качестве последовательных решателей используются Minisat~\cite{bib:minisat},
MapleCOMSPS~\cite{bib:maplecomsps}, painless-maplecomsps~\cite{bib:painless}. Под последовательным
имеется в виду не тип самого решателя, а невозможность параллельного решения разных задач.
То есть, painless-mcomsps является параллельным решателем, но может решать лишь одну задачу
одновременно. Описываемый сервис нужен именно для того, чтобы обеспечить возможность параллельного 
решения разных задач, что необходимо для реализации параллельных алгоритмов решения на основе 
подхода разделяй-и-властвуй~\ref{overview:dac}.

Схема наследования интерфейса \textsc{SolverService} проиллюстрирована на рисунке~\ref{arch:solver:serv-img}.
Описание методов интерфейса приведено в таблице~\ref{arch:solver:serv-def}.

\begin{figure}[H]
    \caption{Интерфейс \textsc{SolverService} и его реализация}
    \centering
    \includegraphics[width=\textwidth]{arch-solver}
    \label{arch:solver:serv-img}
\end{figure}

\begin{table}[H]
    \caption{Описание методов \textsc{SolverService}}\label{arch:solver:serv-def}
    \centering
    \begin{tabularx}{\textwidth}{|*{3}{>{\centering\arraybackslash}X|}}\hline
        Метод & Параметры & Описание \\\hline
        \textsc{solve} & Подстановка, ограничение по времени, callback-функция, вызываемая 
                        при окончании решения & Возвращает объект \textsc{std::future} типизированный
                        результатом решения, и добавляет задачу в очередь \\\hline
        \textsc{load\_problem} & -- & Загружает формулу в решатель \\\hline
        \textsc{interrupt} & -- & Прерывает процесс решения \\\hline
    \end{tabularx}
\end{table}

Реализация \textsc{SequentialSolverService} управляет набором последовательных решателей, работающих
параллельно: обеспечивает их задачами, а также, при необходимости, обеспечивает обмен знаниями.
Для распределения задач реализована безопасная для использования в многопоточной среде очередь.
Для обмена знаниями используются механизмы, реализованные в интерфейсе \textsc{painless} и адаптированные
для использования разных решателей и схем обмена. Схема работы \textsc{SequentialSolverService}
приведена на рисунке~\ref{arch:solver:seq-img}.

\begin{figure}[H]
    \caption{Схема работы \textsc{SequentialSolverService}}
    \centering
    \includegraphics[width=\textwidth]{arch-seq}
    \label{arch:solver:seq-img}
\end{figure}

\section{Использование вероятностных лазеек}\label{arch:solve}

В данной главе описаны реализованные подходы к использованию вероятностных лазеек для решения
задач булевой выполнимости. Описаны \textit{прямой} и \textit{рекурсивный} подоходы к решению,
а также схема с \textit{использованием нескольких лазеек}.

\section{Описание реализации}\label{arch:impl}

В данной главе описана структура исходного кода алгоритма, перечислены возможные опции как при
сборке, так и при запуске приложения.

\chapterconclusion


\chapter{Практическое исследование}\label{research}

В данной главе описаны практические исследования, возникшие в процессе разработки алгоритма.
В разделе~\ref{research:prop} изложены эксперименты касательно эффективности предлагаемых
алгоритмов вывода последствий. В разделе~\ref{research:evol} --- эксперименты, касающиеся
эволюционных алгоритмов, описанных в разделе~\ref{arch:rbs:schema}. Также в разделе~\ref{research:final}
проведены замеры как базового решателя \textsc{painless-mcomsps}, так и разработанного в данной 
работе. Сделаны выводы о сильных и слабых сторонах полученного решения.

\section{Алгоритмы вывода последствий}\label{research:prop}

\section{Эволюционные алгоритмы}\label{research:evol}

\section{Оценка производительности решателя}\label{research:final}

\chapterconclusion

\tbd{Заключение}


\include{chapters/conclusion}

\printmainbibliography

\end{document}
