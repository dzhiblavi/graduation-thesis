%&latex
\label{introduction}
\startprefacepage

Задача булевой выполнимости является крайне важной NP-полной~\cite{bib:cook-levin} алгоритмической
задачей, так как к ней сводится большое число задач. Среди них:
\begin{itemize}
    \item Проверка моделей, использующийся для формальной верификации параллельных систем с
        конечным числом состояний~\cite{bib:use-mc}.
    \item Восстановление секретного ключа в алгоритме RSA~\cite{bib:use-rsa}.
    \item Разложение булевых матриц~\cite{bib:use-bma}, часто применяемое в рекоммендательных
        системах.
    \item 0-1 задача о рюкзаке (широко известная задача класса NP).
    \item Многие задачи планирования, например, one shop scheduling~\cite{bib:use-ojs}, сводятся к SAT.
    \item \tbd{Больше примеров?}.
\end{itemize}

Для задачи булевой выполнимости, как и для любой NP-полной задачи пока не известно алгоритма,
решающего её за полиномиальное от размера входа время. Однако, спектр применения этой задачи
достаточно широк, чтобы возникала практическая польза от разработки эффективных решателей.

Алгоритмы решения задачи булевой выполнимости глобально делятся на две категории. Первая ---
последовательные решатели, то есть работающие в одном потоке исполнения. Существующие эффективные
и широко распостраненные подходы к построению таких решателей более подробно описаны в разделе~
\ref{overview:sequential}. Вторая --- параллельные решатели, нацеленные на максимальную утилизацию доступных
ресурсов многоядерных и многопроцессорных машин для ускорения процесса решения. Более подробно
они описаны в разделе~\ref{overview:parallel}.

Целью данной работы является построение эффективного параллельного алгоритма решения задачи булевой
выполнимости на основе поиска вероятностных лазеек. В рамках исследований в \tbd{paper} данное семейство
подходов показало положительные результаты, однако остается достаточно большое пространство как для
построения более конкурентоспособной реализации, так и для расширения семейства подходов.

В рамках данной работы поставлены и выполнены следующие задачи:
\begin{itemize}
    \item Разработка эффективного алгоритма поиска вероятностных лазеек на основе подходов,
        изложенных в \tbd{paper}.
    \item Адаптация набора существующих решателей для использования в рамках разрабатываемых
        алгоритмов.
    \item Разработка эффективного параллельного сервиса для решения задач с подстановками,
        допускающий обмен знаниями между решателями.
    \item Разработка схем решения с использованием вероятностных лазеек, как предложенных в \tbd{paper},
        так и альтернативных.
    \item Исследование эффективности реализованных схем и сравнение с существующими
        паралелльными решателями.
\end{itemize}

В главе~\ref{overview} содержатся необходимые понятия и определения, детально описаны поставленные
в данной работе задачи, описан подробный план по их решению. Также рассмотрены существующие
подходы к решению задач булевой выполнимости, как параллельные, имеющие непосредсвенное отношение
к данной работе как конкуренты, так и последовательные, устройство которых необходимо понимать,
так как на определенные свойства таких решателей делается упор в устранении недостатков
предлагаемых алгоритмов.

В главе~\ref{arch} детально изложены предлагаемые подходы к решению подзадач, возникших при
разработке алгоритма. Описана схема поиска вероятностных лазеек, способ разделения задач булевой
выполнимости и его применение в данной работе. Изложен подход к параллельному решению подзадач,
указаны недостатки наивного подхода, предложено и обосновано решение данных недостатков. Далее
описана структура програмного кода разработанного приложения.

В главе~\ref{research} описан процесс оценки качества и производительности полученного семейства
алгоритмов. Предоставлены и разобраны результаты экспериментов как над частями алгоритма, так и
над получившимся решатетем в целом. Сделаны выводы относительно конкурентоспособности рассматриваемых
схем решения.

%\printannobibliography
