%&latex
\label{introduction}
\startprefacepage

Задача булевой выполнимости является крайне важной NP-полной~\cite{bib:cook-levin} алгоритмической
задачей, так как к ней сводится большое число задач. Среди них:
\begin{itemize}
    \item Проверка моделей, использующийся для формальной верификации параллельных систем с
        конечным числом состояний~\cite{bib:use-mc}.
    \item Восстановление секретного ключа в алгоритме RSA~\cite{bib:use-rsa}.
    \item Разложение булевых матриц~\cite{bib:use-bma}, часто применяемое в рекоммендательных
        системах.
    \item 0-1 задача о рюкзаке (широко известная задача класса NP).
    \item Многие задачи планирования, например, one shop scheduling~\cite{bib:use-ojs}, сводятся к SAT.
\end{itemize}

Для задачи булевой выполнимости, как и для любой NP-полной задачи пока не известно алгоритма,
решающего её за полиномиальное от размера входа время. Однако, спектр применения этой задачи
достаточно широк, чтобы возникала практическая польза от разработки эффективных решателей.

Алгоритмы решения задачи булевой выполнимости глобально делятся на две категории. Первая --- 
последовательные решатели, то есть работающие в одном потоке исполнения. Существующие эффективные
и широко распостраненные подходы к построению таких решателей более подробно описаны в Главе~
\ref{overview:sequential}. Вторая --- параллельные решатели, нацеленные на максимальную утилизацию доступных
ресурсов многоядерных и многопроцессорных машин для ускорения процесса решения. Более подробно
они описаны в Главе~\ref{overview:parallel}.

Целью данной работы является построение эффективного параллельного алгоритма решения задачи булевой
выполнимости на основе поиска вероятностных лазеек. В рамках исследований в TBD-REF данное семейство
подходов показало положительные результаты, однако остается достаточно большое пространство как для
построения более конкурентноспособной реализации, так и для расширения семейства подходов.

В рамках данной работы поставлены и выполнены следующие задачи:
\begin{itemize}
    \item Разработка эффективного алгоритма поиска вероятностных лазеек на основе подходов,
        изложенных в TBD-REF.
    \item Адаптация набора существующих решателей для использования в рамках разрабатываемых
        алгоритмов.
    \item Разработка альтернативных предложенным в TBD-REF схем решения с использованием
        вероятностных лазеек.
    \item Исследование эффективности реализованных схем и сравнение с существующими
        паралелльными решателями.
\end{itemize}

В Главе~\ref{overview:terms} содержатся необходимые понятия и определения. В 
Главе~\ref{overview:sequential} поверхностно рассмотрены существующие эффективные и широко
распостраненные подходы к реализации последовательных решателей. В Главе~\ref{overview:parallel}
более подробно описаны популярные и успешные стратегии работы параллельных решателей.

В Главе~\ref{arch:rbs} рассмотрен вопрос поиска вероятностных лазеек. В разделе~\ref{arch:rbs:schema}
построены эффективные эволюционные алгоритмы для решения этой задачи. В разделе~\ref{arch:rbs:prop}
описаны изменения, внесенные в исходный код Minisat, позволяющие существенно ускорить процесс поиска решения.
В Главе~\ref{arch:solver} описан построенный параллельный сервис для решения задач булевой
выполнимости. Описан процесс адаптации и унификации существующих последовательных решателей,
способы обмена знаниями между решателями.
В Главе~\ref{arch:solve} приведено описание и описана реализация как упоминаемых в TBD-REF,
так и новых стратегий использования вероятностных лазеек в решателе.

%\printannobibliography
